\chapter{Conventional Desulfurization Processes}
\thispagestyle{plain}

Hydrotreating (or hydrodesulfurization) and Mercaptan Oxidation (MEROX) are the main processes used to remove sulfur from refinery streams.
\section{Hydrodesulfurization}

Hydrodesulfurization is the most commonly used method in the petroleum industry to reduce the sulfur content of crude oil. In most cases HDS is performed by co-feeding oil and \ce{H2} to a fixed-bed reactor packed with an appropriate HDS catalyst. The standard HDS catalysts are NiMo/\ce{Al2O3} and CoMo/\ce{Al2O3}, but there are many more types available.

\begin{figure}[ht]
\centering
\subfloat[\mbox{Thiols}]{
\includegraphics{Images/Thiol.eps}
\label{fig:thiol}}
\quad
\subfloat[\mbox{Sulfides}]{
{\includegraphics{Images/Sulfide.eps}}
\label{fig:sulfide}}
\quad
\subfloat[\mbox{Disulfides}]{
\includegraphics{Images/Disulfide.eps}
\label{fig:disulfide}}
\quad
\subfloat[\mbox{Thiolanes}]{\makebox[4em]
{\includegraphics{Images/Thiolane.eps}}
\label{fig:thiolane}}
\quad
\subfloat[\mbox{Thiophenes}]{\makebox[5em]
{\includegraphics{Images/Thiophene.eps}}
\label{fig:thiophene}}
\par
\subfloat[\mbox{Benzothiophenes}]{\makebox[9em]
{\includegraphics{Images/Benzothiophene.eps}}
\label{fig:benzothiophene}}
\quad
\subfloat[\mbox{Dibenzothiophenes}]{\makebox[9em]
{\includegraphics{Images/Dibenzothiophene.eps}}
\label{fig:dibenzothiophene}}
\quad
\subfloat[\mbox{Benzonaphtothiophenes}]{\makebox[9em]
{\includegraphics{Images/Benzonaphtothiophene.eps}}
\label{fig:benzonaphtothiophenes}}
\caption{Important classes of sulfur-containing compounds in crude oil (R$=$alkyl)}
\label{fig:compounds}
\end{figure}

\begin{figure}
\centering
\fbox{\includegraphics[width=\linewidth]{Images/HDSreactions.eps}}
\caption{Typical hydrodesulfurization reactions \citep{moulijn2001chemical}}
\label{fig:hdsreactions}
\end{figure}

The most difficult sulfur compounds to eliminate by hydrogen treatment belong to the alkyl dibenzothiophene (DBT) family. 4,6-dimethyldibenzothiophene (4,6-DMDBT) is often considered to be the representative molecule of refractory compounds because it is the most abundant. However, other kinds of substituted DBT such as 4,6-diethyldibenzothiophene (4,6-DEDBT) or 4,6-diisopropyldibenzothiophene (4,6-DiPrDBT) contribute together at least to an equivalent part of the S remaining content \citep{Breysse2003129}. Indeed, many lab-scale studies on novel HDS catalysts utilize solutions of compounds like DBT and 4,6-DMDBT as feed constituents to simulate sulfur containing fuels \citep{Gupta2016246,Fraile2016680,Souza2015217,SilvaRodrigo20152,Wang2009206}. Fig. \ref{fig:hdsreactions} shows typical reactions that occur during hydrotreating of sulfur-containing feeds.

\nomenclature{DBT}{Dibenzothiophene}%
\nomenclature{4,6-DMDBT}{4,6-dimethyldibenzothiophene}%
\nomenclature{4,6-DEDBT}{4,6-diethyldibenzothiophene}%
\nomenclature{4,6-DiPrDBT}{4,6-diisopropyldibenzothiophene}%

\section{Selection of catalysts and reactors for hydrotreatment}

The performance of hydroprocessing units is influenced by the selection of the catalysts and the type of reactor to suit a particular feed. Catalysts ranging widely in chemical composition and physical properties are available commercially. Several types of reactors are proven commercially as well. The catalysts and reactors selected for light feeds differ significantly from those selected for heavy feeds. Fixed-bed reactors have been traditionally used for light feeds.

The selection of catalysts must take into account the properties of the feed to be hydroprocessed. The values in Table \ref{tab:crudecompositions} illustrate the differences in the properties of various crudes. These differences influence refining schemes in the refinery. Yields of the distillate fractions to be refined, as well as the residues to be upgraded, determine the amount of available feeds and/or the hydroprocessing capacity requirements.

\begin{table}[htbp]
  \centering
	\caption{Different residue compositions and physical properties \citep{Rana20071216}}
	\label{tab:crudecompositions}
    \begin{tabular}{lcc}
    \toprule
    \textbf{Crude oil} & \textbf{Gravity (\textdegree API)} & \textbf{S (wt.\%)} \\
    \midrule
    Alaska, north slope & 14.9 & 1.8 \\
    Arabian, safaniya & 13   & 4.3 \\
    Canada, Athabasca & 5.8  & 5.4 \\
    Canada, Cold Lake & 6.8  & 5 \\
    California, Hondo & 7.5  & 5.8 \\
    Iranian & --   & 2.6 \\
    Kuwait, Export & 15   & 4.1 \\
    Mexico, Maya & 7.9  & 4.7 \\
    North Sea, Ekofisk & 20.9 & 0.4 \\
    Venezuela, Bachaquero & 9.4  & 3 \\
    \bottomrule
    \end{tabular}%
\end{table}%

The values in Table \ref{tab:crudecompositions} indicate a significant difference in the content of heteroatoms among the chosen crudes. This suggests that a universal catalyst or a catalytic system suitable for hydroprocessing of feeds derived from various sources does not exist.

The selection of catalysts is application dependent. NiMo-catalysts are more hydrogenating, whereas CoMo-catalysts are better at hydrogenolysis \citep{Topsoe}. Hence, CoMo catalysts are preferred for the HDS of unsaturated hydrocarbon streams, whereas NiMo catalysts are preferred for fractions requiring extreme hydrogenation. High asphaltene and high metal content feeds are processed using moving-bed or fluidized bed reactors. Multi-reactor systems consisting of moving-bed and/or ebullated bed reactors in series with fixed-bed reactors can be used to process difficult feeds. For heavy, viscous feeds, the physical properties, shape and size of the catalyst particles become crucial parameters.

Although different types of reactor designs are marketed, they all work on the same principle--all processes use the reaction of hydrogen with the hydrocarbon feedstock to produce \ce{H2S} and a desulfurized hydrocarbon product. The reaction temperature is typically on the order of 290--450\textcelsius{} with a hydrogen gas pressure on the order of 250 and 3000 psi -- the low temperature minimizes cracking reactions \citep{Speight201369}. In some designs, the feedstock is heated and then mixed with the hydrogen rather than the option of passing moderately heated hydrogen into the reactor. The gas mixture is then led over the catalyst bed. The reactor effluent is then cooled, and the oil feed and gas mixture are then separated in a stripper column. Part of the stripped gas may be recycled to the reactor.

\begin{figure}[htbp]
\centering
\fbox{\includegraphics[width=\linewidth]{Images/Hydroprocessingreactors.png}}
\caption{Simplified features of catalytic reactors for upgrading heavy feeds \citep{tagkey2007217}}
\label{fig:hdsreactors}
\end{figure}

\subsection{Downflow Fixed-Bed (Trickle bed) reactor}

Fixed bed reactors are typically used in the HDS of distillates. The feed stock enters at the top of the reactor and the product leaves at the bottom. The catalyst remains in a stationary position with hydrogen and petroleum feedstock passing in a down flow direction through the catalyst bed. The HDS reaction is exothermic and the temperature rises from the inlet to the outlet of each catalyst bed. 
The reaction mixture can be quenched with cold recycled gas at intermediate points in the reactor system. This is achieved by dividing the catalyst charge into a series of catalyst beds and the effluent from each catalyst bed is quenched to the inlet temperature of the next catalyst bed.

The extent of desulfurization is controlled by raising the inlet temperature to each catalyst bed to maintain constant catalyst activity over the course of the process. Fixed-bed reactors are mathematically modeled as plug-flow reactors. The first catalyst bed is poisoned with vanadium and nickel at the inlet to the bed and is a cheaper catalyst (Guard bed). As the catalyst is poisoned in front of the bed, the temperature exotherm moves down the bed and the activity of the entire catalyst charge declines thus requiring a rise in the reactor temperature over the course of the process sequence.

Although fixed-bed HDS units are generally used for distillate HDS, they may also be used for heavy feedstock HDS with special precautions in processing. Heavy feedstock must undergo two-stage electrostatic desalting so that salt deposits do not plug the inlet to the first catalyst bed and the heavy feedstock must be low in vanadium and nickel content to avoid plugging the beds with metal deposits. Hence heavy feedstock HDS reactors require the use of a guard bed. Some of the commercial processes employing fixed bed reactors include:
\begin{itemize}
\item Unibon process
\item HYVAHL process
\item Atmospheric residue desulfurization (ARDS) process
\end{itemize}

\subsection{Moving bed reactors}

\begin{figure}[htbp]
\centering
\subfloat[\mbox{Bunker Reactor}]{
\includegraphics[scale=0.5]{Images/Bunkerreactor.png}
\label{fig:bunker}}
\quad
\subfloat[\mbox{QCR Reactor}]{
{\includegraphics[scale=0.5]{Images/QCRreactor.png}}
\label{fig:qcr}}
\caption{Schematics of moving bed reactors used for hydroprocessing \citep{tagkey2007217}}
\label{fig:reactors}
\end{figure}

Expanded or moving bed reactors are used for very heavy, high metals and/or dirty feedstocks having extraneous fine solid material. They operate in such a way that the catalyst is in an expanded state so that the extraneous solids pass through the catalyst bed without plugging. They are isothermal, which conveniently handles the high exothermicity associated with high hydrogen consumption. Since the catalyst is in an expanded state of motion, it is possible to  withdraw and add catalyst during operation.

Several moving bed catalyst reactors have reached a commercial scale. Among these, bunker reactor and Quick Catalyst Replacement (QCR) reactor are shown in Fig. \ref{fig:bunker} and \ref{fig:qcr} respectively. Compared with fixed bed reactors, the problems associated with pressure drops are not present in moving bed reactors. This allows the use of catalyst particles varying widely in size and shape. Moreover, inorganic solids present in heavy feeds move through the reactor together with the catalyst and exit at the bottom of the reactor with the spent catalyst. Therefore, processes employing moving bed reactors do not require a guard chamber.

\begin{table}[t]
\centering
\caption{Yields and properties of products from different reactors}
\label{tab:reactoryields}
\begin{tabular}{@{}llll@{}}
\toprule
                        & \textbf{Fixed/moving} & \textbf{Ebullated} & \textbf{Slurry} \\ \midrule
\textbf{Naphtha}        &                       &                    &                 \\
Yield/feed, wt\%        & 1--5                   & 5--15               & 10--15           \\
Density, kg/L           & 0.71--0.74             & 0.71--0.72          & 0.72            \\
sulfur, wt\%            & \textless0.01         & 0.01--0.2           & 0.06            \\
Nitrogen, ppm           & \textless20           & 50--100             & 200             \\
\textbf{Gas oil}        &                       &                    &                 \\
Yield/feed, wt\%        & 10--25                 & 20--30              & 40--45           \\
Density, kg/L           & 0.850--0.875           & 0.840--0.860        & 0.866           \\
sulfur, wt\%            & \textless0.1          & 0.1--0.5            & 0.7             \\
Nitrogen, ppm           & 300--1200              & \textgreater500    & $\sim$ 1800           \\
\textbf{Vacuum gas oil} &                       &                    &                 \\
Yield/feed, wt\%        & 20--35                 & 25--35              & 20--25           \\
Density, kg/L           & 0.925--0.935           & 0.925--0.970        & 1.01            \\
sulfur, wt\%            & 0.25--0.50             & 0.5--2.0            & 2.2             \\
Nitrogen, ppm           & 1500--2500             & 1600--4000          & 4300            \\
\textbf{Vacuum residue} &                       &                    &                 \\
Yield/feed, wt\%        & 30--60                 & 15--35              & 10--20           \\
Density, kg/L           & 0.990--1.030           & 1.035--1.100        & 1.16            \\
sulfur, wt\%            & 0.7--1.5               & 1--3                & 2.7             \\
Nitrogen, ppm           & 3000--4000             & \textgreater3300   & 11              \\
Asphaltenes (heptane)   & 5--10                  & \textgreater20     & 26              \\ \bottomrule
\end{tabular}
\end{table}

The feasibility of a process employing moving bed reactors compared with fixed bed reactors is affected by the capital cost of reactor systems, although this may be offset by the lower relative catalyst consumption. For example, additional high-pressure vessels upstream and downstream of bunker or QCR reactors are required which add to the capital cost. Also, the configurations of moving bed reactors are more complex than that of the fixed bed reactors. Therefore, design of fixed bed reactors is much more simple. In addition, different yield and quality of products are obtained from different reactors.

\subsection{Ebullated bed Reactors}

Ebullated bed reactors are designed to handle the most problematic feeds such as VRs and heavy crudes containing high contents of metals, asphaltenes, sediments as well as dispersed clay and minerals. The most important feature of such reactors is their capability to periodically withdraw and add catalyst to the reactor without interrupting operation \citep{Furimsky1998177}. The bed design ensures ample free space between particles allowing entrained solids to pass through the bed without accumulation, plugging, or increased pressure drop. This allows utilization of catalyst particles smaller than 1 mm in diameter, resulting in a significant increase in reaction rate due to reduced diffusion limitations.

\nomenclature{VR}{Vacuum Residue}

\section{Mercaptans Removal}
\begin{figure}[htbp]
\centering
\fbox{\includegraphics[scale=0.53]{Images/MEROX.png}}
\caption{Typical MEROX flowsheet \citep{Fahim2010377}}
\label{fig:merox}
\end{figure}
The predominant sulfur compounds in refinery products that usually have an unpleasant smell are mercaptans. They are corrosive and disturb fuel stability due to gum formation. The principle of mercaptans removal is oxidation (MEROX). MEROX sweetening involves the catalytic oxidation of mercaptans to disulfides in the presence of oxygen and alkalinity. Air provides the oxygen, while caustic soda provides the alkalinity. Oxygen reacts with mercaptans through the following reaction:
\begin{center}
4\ce{RSH} + \ce{O2} $\longrightarrow$ \ce{RSSR} + 2\ce{H2O}
\end{center}
Removal of mercaptans by extraction starts with dissolving them in caustic soda based on the following reaction:
\begin{center}
\ce{RSH} + \ce{NaOH} $\rightleftharpoons$ \ce{NaSR} + \ce{H2O}
\end{center}
The equilibrium occurs between the RSH organic phase and the RSH that dissolves in the aqueous phase.

The rich caustic soda containing the extracted mercaptans in the form of mercaptides is regenerated as shown in the following reaction:
\begin{center}
4\ce{NaSR} + \ce{O2} $\rightleftharpoons$ 2\ce{H2O} + 2\ce{RSSR} + 4\ce{NaOH}
\end{center}
Fig. \ref{fig:merox} shows the typical flowsheet of a MEROX plant. The MEROX unit consists of a fixed-bed reactor followed by a caustic settler. Air, the source of oxygen, is injected into the feedstock upstream of the reactor. The operating pressure is chosen to assure that the air required for sweetening in completely dissolved at the operating temperature. The sweetened stream exits the reactor and flows to the caustic settler. The caustic soda settler contains a reservoir of caustic soda for use in keeping the MEROX catalyst alkaline. The solvent product leaving the water wash flows to a sand filter containing a simple bed of coarse sand that is used to remove free water and a portion of the dissolved water from the product. The regenerated caustic soda is recycled to the
MEROX reactor.
