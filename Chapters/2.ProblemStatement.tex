\chapter{Problem Statement}
\thispagestyle{plain}


\textbf{Indices} \\
$i$ = tasks \\
$j$ = units  \\
$s$ = states  \\
$u$ = utilities \\
\textbf{Sets} \\
$\mathcal{J} = $ Set of available processing tasks \\
$\mathcal{S} = $ Set of states (materials) \\
$\mathcal{S} = $ Set of utilities \\
$\mathcal{I} = $ Set of processing tasks \\
$\mathcal{I}_j = $ Set of processing tasks that can be performed in unit $j$\\
$\mathcal{I}_s^p = $ Set of tasks that produce state $s$\\
$\mathcal{I}_s^c = $ Set of tasks that consume state $s$\\
$\mathcal{I}_u = $ Set of tasks that consume utility $u$\\
$\mathcal{I}^{zw} = $ Set of tasks that produce a zero-wait state \\
\textbf{Parameters} \\
$\alpha_i = $ Fixed processing time of task $i$ \\
$\beta_i = $ Processing time of task $i$ per unit batch \\
$\gamma_{iu} = $ Fixed consumption of utility $u$ by task $i$ \\
$\delta_{iu} = $ Consumption of utility $u$ by task $i$ per unit batch \\
$\rho_{is} = $ proportion of state $s$ in the total production/consumption by task $i$ \\
$B_i^{\text{min}} = $ minimum capacity for task $i$ in unit $j$ \\
$B_i^{\text{max}} = $ maximum capacity for task $i$ in unit $j$ \\
$U_u^{\text{max}} = $ maximum availability of utility $u$ \\
$S_{s0} = $ Initial amount of state $s$ \\
$S_{s}^{\text{max}} = $ maximum storage capacity for state $s$ \\
$D_s = $ Demand of state $s$ at the end of horizon \\
$P_s = $ Price of state $s$ \\
$H = $ Horizon

Given the above data for units, states, utilities, tasks, horizon and demands, a feasible schedule has to be obtained in order to either maximize profit within horizon $H$ or minimize the time taken to produce demands $D_s$ for states $s$.

