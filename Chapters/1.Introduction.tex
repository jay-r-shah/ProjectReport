\chapter{Introduction}
\thispagestyle{plain}

Scheduling is a decision-making process that plays an important role in most manufacturing and service industries. %\ref{}. 
Scheduling problems arise in almost any type of industrial production facilities where given tasks need to be processes using specified resources. In a chemical process, production must be planned such that equipment, material and utilities are available at the manufacturing facility when they are needed to realize the production tasks. Production scheduling comprises the activity of planning in detail the production of a product or products in a given production facility. It boils down to the following main decisions \citep{HARJUNKOSKI2014161}:
\begin{itemize}
\item What production tasks to execute?
\item Where to process the production tasks?
\item In which sequence to produce?
\item When to execute the production tasks?
\end{itemize}

For batch processes, short-term scheduling deals with the allocation of a set of limited resources over time to manufacture one or more products following a batch recipe \citep{MENDEZ}. There has been significant development of optimization approaches to scheduling over the last two decades. The first mathematical programming approach the scheduling of multi-purpose, multi-product batch plants was proposed by \cite{KONDILI1993211}. This approach introduces the state task network (STN) representation where the process is described as a bipartite graph consisting of states and tasks.