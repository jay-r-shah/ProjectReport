\chapter{Novel desulfurization processes}
\thispagestyle{plain}

Hydrodesulfurization in combination with carbon rejection technologies, such as coking and fluid catalytic cracking (FCC) are the main technologies industrially employed for the desulfurization of crude. Although these technologies are quite capable of desulfurizing heavy oil, their carbon footprints are substantial since all of these technologies, including the production of hydrogen that is needed for HDS, involve high-temperature processing. The refining cost increases as heavier and sulfur-rich oils are being processed. Hence, alternative desulfurization pathways are of interest. In this chapter, a few non-conventional desulfurization techniques have been described.

\section{Extraction by ionic liquids}

HDS is limited due to the low conversion rates of the higher aromatics. The low sulfur limits can only be met by extreme operating conditions in terms of pressure and residence time \citep{C1GC15196G}. Hence, the process becomes uneconomic and energy inefficient. Further, in order to fulfill gasoline sulfur limits, deep desulfurization of all streams contributing to the gasoline-pool has to carried out. Although in comparison to sulfur components in diesel fuel, HDS of thiophene is reached more easily; the bottleneck of this process is the decrease in octane number due to simultaneous hydrogenation of the olefins present in the stream. In the case of diesel fuels, due to the severe HDS conditions that are necessary to produce ultra-low sulfur diesel, the cetane number is affected as well.

Liquid-liquid extraction is a technique which has been proposed for deep desulfurization because of its simplicity an and mild operating conditions. Extraction desulfurization with Ionic Liquids (ILs) as extraction solvents has the potential for alternative and future complementary technology for deep desulfurization. An ionic liquid is a non-volatile organic liquid salt, which potentially can extract sulfur and also organic nitrogen compounds in fuels by virtue of its polarity. \cite{Ito2006446} concluded that their application for desulfurization is limited due to the co-extraction of aromatic hydrocarbons. However, due to stricter environmental regulations, the aromatic hydrocarbon content (along with sulfur content) also needs to be reduced. Therefore, the co-extraction of aromatics may be an advantage since this allows for removal of both types of compounds in a single step.

\begin{figure}[t]
\centering
\fbox{\includegraphics[width=0.8\linewidth]{Images/Extractionscheme.png}}
\caption{Concept of deep desulfurization of refinery streams by extraction
with ILs \citep{B407028C}}
\label{fig:extraction}
\end{figure}

\nomenclature{IL}{Ionic Liquid}

ILs have excellent extraction properties for organic S- and N-compounds and are -- if chosen carefully -- insoluble in oils. The basic concept of such an extraction process is illustrated in Fig. \ref{fig:extraction}. \cite{C1GC15196G} have characterized the capacity and suitability of several IL solvents for thiophene and dibenzothiophene. For both sulfur aromatics, pyridinium-based ionic liquids [3-mebupy]N\ce{(CN)2} and [4-mebupy]N\ce{(CN)2}, [4-mebupy]SCN are suitable candidates since the capacity, as well as the selectivity, are higher than those of sulfolane.\footnote{Sulfolane is the commercial extraction solvent with the highest aromatic capacity, mostly used for aromatics extraction from different petroleum fractions. Hence results of ionic liquids are compared to sulfolane.} In case of the investigated imidazolium-based ionic liquids, only [BMIM]C\ce{(CN)3} fulfils the criteria and is superior to sulfolane. [BMIM]N\ce{(CN)2} only has a higher selectivity, while the selectivity of [BMIM]SCN is comparable to that of sulfolane.

Extractive desulfurization becomes increasingly difficult and unselective as the heaviness of the oil increases. Solvent loss and recovery are important detractors when desulfurizing heavy oil. The sulfur compounds are high
boiling and the heavy oil is viscous. It is unlikely that a solvent can be found that will be sulfur-selective based purely on a physical extraction. It is anticipated that any breakthrough in extractive desulfurization of heavy oil will, out of necessity, be in reactive extractive desulfurization, i.e. a solvent that chemically reacts with sulfur in sulfur-containing compounds to produce a separate phase \citep{Javadli}. Even so, this does not eliminate the problems associated with solvent recovery, which must still be addressed.

\section{Biodesulfurization}

\nomenclature{BDS}{Biodesulfurization}
\nomenclature{ODS}{Oxidative desulfurization}

Biodesulfurization is a non-invasive approach that can specifically remove sulfur from refractory hydrocarbons under mild conditions and it can be potentially used in industrial desulfurization. Intensive research has been
conducted in microbiology and molecular biology of the competent strains to increase their desulfurization activity; however, even the highest activity obtained is still insufficient to fulfill the industrial requirements. 

Sulfur forms 0.5--1\% of bacterial cell dry weight. Microorganisms require sulfur for their growth and biological activities. Sulfur generally occurs in the structure of some enzyme cofactors (such as Coenzyme A, thiamine and biotin), amino acids and proteins (cysteine, methionine, and disulfur bonds). Microorganisms, depending on their enzymes and metabolic pathways, may have the ability to provide their required sulfur from different sources. Some microorganisms can consume the sulfur in thiophenic compounds such as DBT and reduce the sulfur content in fuel. Desulfurization by microorganisms is potentially advantageous. Firstly, it is carried out in mild temperature and pressure conditions; therefore, it is considered as an energy-saving process (an advantage over HDS). Secondly, in biological activities, biocatalysts (enzymes) are involved; therefore, the desulfurization would be highly selective (an advantage over ILs).

\cite{Soleimani2007570} have described three main types of biodesulfurization:
\begin{itemize}
\item Destructive biodesulfurization
\item Anaerobic biodesulfurization
\item Specific oxidative desulfurization
\end{itemize}

Aerobic BDS was proposed as an alternative to hydrodesulfurization of crude oil. It was reported that BDS by \emph{Pantoea agglomerans} D23W3 resulted in 61\% sulfur removal from a light crude oil that originally contained 0.4\% sulfur and 63\% sulfur removal from a heavy crude oil that originally contained 1.9\% sulfur. It was found that integrated methods performed better than just BDS. By combining ODS with BDS it was possible to achieve 91\% sulfur removal from heavy oil \citep{agarwalsharma}.

The main reasons that BDS is not commercially employed for crude oil desulfurization are the low activity, the logistics of sanitary handling, shipment, storage and use of miccroorganisms within the refinery environment.

\section{Olefin alkylation of thiophenic sulfur}

The OATS process, first developed by British Petroleum, can be seen as a good alternative for the conventional desulfurization because of the comparative advantages of mild reaction conditions, no need for other reactants and minimal loss of octane number. Typically, this process consists of two steps: first, thiophene and its derivates are alkylated with the olefins in gasoline over some acidic catalysts; second, the alkylated sulfur-containing compounds are separated from gasoline by distillation. The alkylation technique is based on the concept that, when the boiling temperature of organosulfur compounds is shifted to a higher value, they can be removed from light fractions by distillation and concentrated in the heavy boiling part of the refinery streams. 

\nomenclature{OATS}{Olefin alkylation of thiophenic sulfur}

\begin{figure}[htbp]
\centering
\fbox{\includegraphics[width=\linewidth]{Images/OATS.eps}}
\caption{OATS process \citep{el2015handbook}}
\label{fig:OATS}
\end{figure}

In conventional OATS technologies, the separations and reactions are carried out with different equipment as shown in Fig. \ref{fig:OATS}. From a process intensification point of view, catalytic distillation may be more effective and economical. In the process of catalytic distillation, separation and catalytic reaction occur in the same vessel, so the conversion of the reactants can be greatly enhanced.

Using equilibrium steady state simulations, \cite{JCTB:JCTB4604} have presented a sensitivity analysis and economic evaluation of the catalytic distillation process for alkylation desulfurization of FCC gasoline.  The results indicated that the operating pressure impacts both the separation and reaction; hence increasing the operating pressure is not always beneficial to the catalytic distilation process. However, a high feed pressure is an economical option despite it having no significant effect on sulfur transfer.

A major advantage of this process is that less hydrogen is consumed since only a relatively low volume of the naphtha stream is hydrotreated. One of the disadvantages of the process is that the alkylated sulfur compounds produced require more severe hydrotreating conditions to eliminate sulfur \citep{el2015handbook}.

%\section{Oxidative desulfurization}